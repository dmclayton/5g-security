%
% File          : 5GAKA-proof.tex
% Description   : 5G-AKA Proof Paper
% Authors       : Clayton & Harmon
%
% Last Modified : Fri Dec  1 08:17:28 EST 2000
%
% BDOC PARAM offset=0in,0in

% Page style
\documentclass[11pt, pdftex]{article}
\usepackage{epsf}
\usepackage{epsfig}
\usepackage{times}
\usepackage{ifthen}
\usepackage{comment}

\usepackage[margin=1in]{geometry}


\title{5G-AKA: A Formal Verification}
\author{David Clayton and Ira Harmon}
\date{November 29, 2018}


\begin{document}
\maketitle
\textbf{Abstract:} 

\newpage
\section{Introduction}
By 2019 over 5 Billion people are expected to own a mobile phone.  Currently over 62 percent of the world population uses mobile phones.  As cell phones become more pervasive their use touches every aspect of modern life: Facebook updates, news, and banking transactions are all increasingly done via cell.  At this critical time in the evolution of cellular technology, 3GPP, the body that standardizes cell phone protocols is preparing to deploy 5G.  And while 5G promises to connect more users with better service than previous generations, the unrestrained growth of the technology makes the security implications of 5G critical.  5G-AKA (Authentication and Key Agreement) is the first line of defense in securing mobile users communications.  The protocol authenticates the user and distributes information that is used to derive session keys.        

\nocite{*}
\bibliography{References}
\bibliographystyle{plain}
\end{document}
